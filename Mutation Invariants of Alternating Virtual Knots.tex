\documentclass[12pt]{report}

%%%%%%%%%%%%%%%%%%%%%%%%%%%%%%%%%%%%%%%%%%%%%%%%%%%%%%%%

%%% General Packages
\usepackage{amsmath, amssymb, amsthm}
\usepackage{titling}
\usepackage{titlesec}
\usepackage{geometry}
\usepackage{enumerate}

\usepackage[hidelinks]{hyperref}



%%% Font and Text Packages
\usepackage{newpxtext}
\usepackage{newpxmath}

\usepackage[dvipsnames]{xcolor}


%%% Graphics, Figure and Listing Packages
\usepackage{graphicx}
\usepackage{float}
\usepackage{calc}
\usepackage{caption}
\usepackage{subcaption}


%%% Epigraph
\usepackage{epigraph}


%%% Listings
\usepackage{listings}
\usepackage{lstfiracode}


%%% Bibliography Packages
\usepackage[style=alphabetic]{biblatex}
\bibliography{references}

%%%%%%%%%%%%%%%%%%%%%%%%%%%%%%%%%%%%%%%%%%%%%%%%%%%%%%%%

%%% Page Formatting Options
\geometry{left = 2.5cm}
\geometry{right = 2.5cm}
\geometry{top = 2.5cm}
\geometry{bottom = 2.5cm}

%%% Section and Chapter Titling Options

\titleformat{\chapter}[display]
	{\normalfont\bfseries\LARGE}
	{\chaptertitlename~\thechapter}
	{0pc}
	{{\color{black!30!white}\titlerule[2pt]}\vspace{0.8pc}\normalfont\Large}
  
\titleformat{name=\chapter, numberless}[display]
	{\normalfont\bfseries\LARGE}{}{1pc}
	{\normalfont\Large}
  
\titlespacing*{\chapter}{0pt}{30pt}{40pt}

\titleformat{\section}
	{\normalfont\bfseries\large}
	{\normalfont\bfseries\large{\thesection}}
	{1em}
	{}

%%% Hyperlink Formatting Options
\hypersetup{
    colorlinks,
    linkcolor={black},
    citecolor={blue!60!black},
    urlcolor={blue!80!black}
}

%%% Epigraph Options and Setup
\setlength\epigraphwidth{0.6\textwidth}
\setlength{\epigraphrule}{0pt}


%%%%%%%%%%%%%%%%%%%%%%%%%%%%%%%%%%%%%%%%%%%%%%%%%%%%%%%%

%%% Graphics and Figure Options
% Graphics path (necessary for .svg images).
\graphicspath{{graphics/}}
\counterwithout{figure}{chapter}

% Caption setup
\captionsetup{margin=1.5cm}


%%% Listings Options
\definecolor{codegreen}{rgb}{0,0.6,0}
\definecolor{codegray}{rgb}{0.5,0.5,0.5}
\definecolor{codepurple}{rgb}{0.58,0,0.82}
\definecolor{codeback}{rgb}{0.95,0.95,0.92}
\lstset{
	language=Python,
	backgroundcolor=\color{codeback},   
	commentstyle=\color{codegreen},
	keywordstyle=\color{magenta},
	numberstyle=\tiny\color{codegray},
	style=FiraCodeStyle,   % Use predefined FiraCodeStyle
	basicstyle=\ttfamily,   % Use \ttfamily for source code listings
	numbers=left
}

%%%%%%%%%%%%%%%%%%%%%%%%%%%%%%%%%%%%%%%%%%%%%%%%%%%%%%%%

%%% Personal Macros
\newcommand{\N}{\mathbb{N}}
\newcommand{\R}{\mathbb{R}}
\newcommand{\Z}{\mathbb{Z}}
\newcommand{\T}{\mathbb{T}}
\renewcommand{\S}{\mathbb{S}}
\newcommand{\ip}[2]{\langle #1, #2 \rangle}

\newcommand{\kob}{\operatorname{kob}}

%%% Drafting Macros
\newcommand{\notered}[1]{{\color{Red} \textbf{#1}}}
\newcommand{\notegreen}[1]{{\color{Green} \textbf{#1}}}
\newcommand{\noteblue}[1]{{\color{Blue} \textbf{#1}}}

%%% amsthm Options
\newtheorem*{theorem}{Theorem}
\newtheorem*{proposition}{Proposition}
\newtheorem*{conjecture}{Conjecture}
\newtheorem*{lemma}{Lemma}
\newtheorem*{corollary}{Corollary}


%%%%%%%%%%%%%%%%%%%%%%%%%%%%%%%%%%%%%%%%%%%%%%%%%%%%%%%%

\begin{document}
	
%%% Make titlepage.

% Titlepage Options
\author{Damian Lin}
\title{Mutation Invariants of Alternating Virtual Knots}

\cleardoublepage \thispagestyle{empty}
\null \vfil
\begingroup
\LARGE \bfseries \centering
\openup \medskipamount
\thetitle \par \vspace{30pt}
\centering \mdseries \theauthor \par \bigskip
\endgroup
\vfil \vfil \vfil
\begin{center}
	An essay submitted in partial fulfilment of\\
	the requirements for the degree of\\
	Bachelor of Science/Bachelor of Advanced Studies (Honours)
	\vfil\vfil
	{\large Pure Mathematics\\[5pt]
		University of Sydney}\\
	\vskip6mm
	\includegraphics[width=25mm]{graphics/USY_MB1_CMYK_Stacked_Logo}
	\vfil
	\normalsize\today
\end{center}
\vfil
\cleardoublepage

\tableofcontents

\chapter*{Introduction}
\addcontentsline{toc}{chapter}{Introduction}

\notegreen{Introduction...}

\notered{very draft}
In order to keep this text relatively self-contained, we begin in Chapter 1 with the study of knots, for without the bread and butter of knot theory, knot invariants. This provides some motivation for the rest of this chapter in which we slowly build our way up to a specific invariant of alternating knots, the lattice of integer flows of the Tait graph, closely following the work of Greene \cite{lattices-graphs-mutation}. We then look at the formulation given by Greene, the $d$-invariant and how that relates to Heegard-Floer homology and another formulation of the $d$-invariant. In Chapter 2, we examine virtual knots, a generalisation of knots with a several equivalent formulations that allows knots to have diagrams on an orientable surface of any genus. The aim of this is to extend this invariant to the virtual setting, and examine what properties it is able maintain in this new environment. We suspect the virtual $d$-invariant is less powerful than another invariant known as the Gordon-Litherland Linking Form, and we examine their relationship in Chapter $3$, and in Chapter $4$ we discuss how Gauss codes provide a way for computers to deal with knots. In Chapter $5$ we examine an algorithm to compute these invariants, and present a proof that indeed, the virtual $d$-invariant is not as strong as the Gordon-Litherland Linking Form, and therefore not a complete mutation invariant of alternating knots.

\chapter*{Acknowledgements}
\addcontentsline{toc}{chapter}{Acknowledgements}

Thanks to ...

We would like to thank Hans Boden for an insight into the proof of the proposition in Chapter 5.


\chapter{Knots and the Lattice of Integer Flows}
\epigraph{\itshape ``A knot!'' said Alice, ``oh, do let me help to undo it!''\\``I shall do nothing of the sort!'' said the mouse.}{--- Lewis Carroll, \textit{Alice's Adventures in Wonderland}}

\section{Knots and Knot Invariants}

We begin with a swift introduction to the rich and marvellous study of tangled-up pieces of string: the theory of knots. Despite being a complex and intricate field, any child can intuitively grasp the concept of a knot as a closed loop of string in space. To formalise this and remove any pathological examples that are inconsistent with this intuition, we define a \textit{knot} to be an injective embedding of the circle into $3$-space, $K: \S^{1} \lhook\joinrel\longrightarrow \R^{3}$. The requirement that the embedding be injective ensures that the string does not intersect itself. Fig.~\ref{fig:knot-examples} gives a few example knots.

\begin{figure}[hbt!]
	\centering
	\hspace*{\fill}
	\begin{subfigure}[b]{0.3 \textwidth}
		\centering
		\def\svgscale{0.2}
		\input{graphics/unknot.pdf_tex}
		\caption{Unknot or Trivial Knot, $0_{1}$}
		\label{fig:unknot}
	\end{subfigure}
	\hspace*{\fill}
	\begin{subfigure}[b]{0.3 \textwidth}
		\centering
		\def\svgscale{0.2}
		\input{graphics/trefoil.pdf_tex}
		\caption{Trefoil Knot, $3_{1}$}
		\label{fig:trefoil}
	\end{subfigure}
	\hspace*{\fill}
	\begin{subfigure}[b]{0.3 \textwidth}
		\centering
		\def\svgscale{0.2}
		\input{graphics/6-3_knot.pdf_tex}
		\caption{The knot $6_{3}$}
		\label{fig:6-3-knot}
	\end{subfigure}
	\caption{Some examples of knots, presented through knot diagrams. Their names in the Rolfsen knot table are also given.}
	\label{fig:knot-examples}
	\hspace*{\fill} 
\end{figure}

The example in Fig.~\ref{fig:unknot-twisted} shows a knot that can be `untwisted' to look like the example in Fig.~\ref{fig:unknot}. In general, we do not distinguish between knots if there is some way to deform one, without breaking the circle or passing it through itself, into the other. We say two knots $K_{1}$ and $K_{2}$ are \textit{equivalent} or equal if there is an \textit{ambient isotopy} from $K_{1}$ to $K_{2}$; that is, a continuous map
\[F: \R^{3} \times I \longrightarrow \R^{3}\]
such that at each $t \in I$, the corresponding $f_{t}: \R^{3} \longrightarrow \R^{3}$ is a homeomorphism of $\R^{3}$, and $f_{0}$ is the identity map on $\R^{3}$, while $f_{1} \circ K_{1} = K_{2}$.

Though a central objective of knot theory it to classify knots up to ambient isotopy, it can be hard to write down explicit ambient isotopies directly. Rather, we use other means to classify knots which will be the subject of the remainder of this section.

\begin{figure}[hbt]
	\centering
	\hspace*{\fill}
	\begin{subfigure}[b]{0.35 \textwidth}
		\centering
		\def\svgscale{0.2}
		\input{graphics/unknot_twisted.pdf_tex}
		\caption{A less complicated diagram for the unknot.}
		\label{fig:unknot-twisted}
	\end{subfigure}
	\hspace*{\fill}
	\begin{subfigure}[b]{0.35 \textwidth}
		\centering
		\def\svgscale{0.2}
		\input{graphics/unknot_goeritz.pdf_tex}
		\caption{A more complicated diagram for the unknot.}
		\label{fig:unknot-goertiz}
	\end{subfigure}
	\hspace*{\fill}
	\caption{It can be difficult to tell whether two diagrams represent the same knot. The diagram in (b) is from \cite{notes-on-knot-theory}.}
	\label{fig:more-unknots}
\end{figure}

Though we claimed that the objects in Fig.~\ref{fig:knot-examples} were knots, they are really projections of knots onto the plane with markings we call \textit{crossings} to make clear that some \textit{strand} of the knot passes over another strand. We refer to these objects as \textit{knot diagrams}. We consider diagrams equivalent up to planar isotopy as $4$-valent graphs. In other words, diagrams can just as well be thought of as combinatorial objects consisting of a list of crossings with labelled ends, and a a description of how these ends connect. So long as the ends are connected in a planar way, the exact path that connects them is unimportant.

Different diagrams can represent the same knot: for example, the diagrams in Fig.~\ref{fig:unknot} and Fig.~\ref{fig:unknot-twisted}, as the latter could be `untwisted' into the former. It is not always obvious when two diagrams represent the same knot. An example is Fig.~\ref{fig:unknot-goertiz}, yet another diagram for the unknot, but it may not be immediately obvious how this knot can be untangled. The unknot is not special here -- every knot has many diagrams, and there is no upper bound on the number of crossings, as we could continually apply the $R\textit{I}$ Reidemeister move to create diagrams with arbitrarily many crossings for any knot. However, each knot does have a lower bound on the number of crossings any diagram representing it can have, which we call its \textit{crossing number}. For example, the crossing number of the unknot is $0$, and the crossing number of the Trefoil knot is $3$.

Diagrams are easier to work with than embeddings, thanks to the following foundational theorem due to Alexander-Briggs  \cite{types-of-knotted-curves} and independently Kurt Reidemeister \cite{elementary-justification-knot-theory}.
\begin{theorem}(The Reidemeister Theorem)
Two knot diagrams $D_{1}$ and $D_{2}$ represent equivalent knots if and only if there is some sequence of finitely many moves of the types given in Fig.~\ref{fig:reidemeister_moves} that transform $D_{1}$ into $D_{2}$.
\end{theorem}

\begin{figure}[hbt!]
	\centering
	\hspace*{\fill}
	\begin{subfigure}[b]{0.35 \textwidth}
		\centering
		\def\svgscale{0.22}
		\input{graphics/reidemeister_1.pdf_tex}
		\caption{\textit{R}I}
	\end{subfigure}
	\hspace*{\fill}
	\begin{subfigure}[b]{0.35 \textwidth}
		\centering
		\def\svgscale{0.22}
		\input{graphics/reidemeister_2.pdf_tex}
		\caption{\textit{R}II}
	\end{subfigure}
	\hspace*{\fill}
	\\
	\hspace*{\fill}
	\begin{subfigure}[b]{0.35 \textwidth}
		\centering
		\def\svgscale{0.22}
		\input{graphics/reidemeister_3.pdf_tex}
		\caption{\textit{R}III}
	\end{subfigure}
	\hspace*{\fill}
	\caption{The Reidemeister moves.}
	\label{fig:reidemeister_moves}
\end{figure}


The early study of knot theory by pioneers such as P.G. Tait, C.N. Little and T. Kirkman involved trying to find, by hand, sequences of moves to show two knots were equivalent. And in hindsight they did remarkably well with so few tools. But Tait himself noted that it was impossible by these means alone to ever prove that two knots were distinct. The modern way we do this is by using knot invariants. If we define some map from knot diagrams to some other class of objects, perhaps a truth value, a polynomial, or a group, and show that none of the Reidemeister moves changes the value of this map, then we have a well-defined function on knots: a \textit{knot invariant}. This allows us to prove that two knots are different, for if they take on different values under some invariant, they must be distinct. However, invariants are one-sided in nature -- taking different values can tell us that two knots are different, but two knots taking on the same value of some invariant doesn't necessitate that they be equivalent knots. Such an invariant -- one that is an injective function from the class of knots is called a \textit{complete} invariant, and while we now know of a zoo of different invariants, we have yet to find what is perhaps the Holy Grail of knot theory: a complete invariant that is also easy to compute.

\section{Alternating Knots, Knot Mutation and the Tait Graph}
We call a knot diagram \textit{alternating} if, traversing the diagram, the crossings alternate under and over. For every diagram that is alternating, it is possible to construct a non-alternating diagram of an equivalent knot: one simply needs to apply a type \textit{R}I Reidemeister move appropriately to any strand of the knot: for example the alternating diagram of the Trefoil knot in Fig.~\ref{fig:trefoil} transforms into the non-alternating diagram in Fig.~\ref{fig:trefoil-nonalternating}. Hence we define an \textit{alternating knot} as a knot for which can be represented by an alternating diagram. For low crossing-number, that is for knots that can be represented by diagrams with few crossings, many of the knots are alternating, but this trend quickly reverses as crossing number is increased.

An alternating knot diagram is \textit{reduced} if there are no crossings that can be immediately removed by an application of the $R\textit{I}$ Reidemeister move. The diagram in Fig.~\ref{fig:trefoil} is a reduced alternating diagram, while the diagram in Fig.~\ref{fig:trefoil-nugatory} is alternating but not reduced.


\begin{figure}[hbt!]
	\centering
	\hspace*{\fill}
	\begin{subfigure}[b]{0.4 \textwidth}
		\centering
		\def\svgscale{0.23}
		\input{graphics/trefoil_nonalternating.pdf_tex}
		\caption{A non-alternating diagram of the Trefoil knot.}
		\label{fig:trefoil-nonalternating}
	\end{subfigure}
	\hspace*{\fill} \hspace*{\fill}	\hspace*{\fill}
	\begin{subfigure}[b]{0.4 \textwidth}
		\centering
		\def\svgscale{0.23}
		\input{graphics/trefoil_nugatory.pdf_tex}
		\caption{An alternating but not-reduced diagram of the Trefoil knot.}
		\label{fig:trefoil-nugatory}
	\end{subfigure}
	\hspace*{\fill} 
	\caption{Diagrams of the Trefoil knot.}
	\label{fig:trefoil-diagrams}
\end{figure}

Tait made a series of conjectures about alternating knots in his early attempts to tabulate knots. The most important of these conjectures, in the sense that is implies the others, is known as the flyping conjecture, and it relates reduced alternating diagrams of the same alternating knot by moves known as flype moves.

A \textit{tangle} in a knot diagram is a region of the plane that is homeomorphic to a disc, such that the knot crosses the boundary of the disc exactly four times, as in Fig.~\ref{fig:kinoshita-terasaka-knot} and Fig.~\ref{fig:conway-knot}. A \textit{flype} move is a diagrammatic move that flips a tangle but does not alter the knot type, as in Fig.~\ref{fig:flype}. Tait's Flyping conjecture was proven by Menasco and Thistlethwaite in \cite[p. 166]{classification-alternating-links}:

\begin{theorem}[Flyping Theorem]
	Any two minimal crossing (and therefore reduced) alternating diagrams which represent the same knot are related by a sequence of finitely many flypes, as seen in Fig.~\ref{fig:flype}. \notered{I realised in Greene this applies for minimal crossing (not just reduced) which makes sense thinking about flypes. How then do we prove later on that lattice of integer flows is a truly an invariant of alternating knots (why are the lattices the same for diagrams if, say, one is reduced and one is not)?}
\end{theorem}

\begin{figure}[hbt]
	\centering
	\def\svgscale{0.5}
	\input{graphics/flype.pdf_tex}
	\caption{A general flype move.}
	\label{fig:flype}
\end{figure}

This is a Reidemeister-like theorem for alternating knots in that is relates equivalence of diagrams to the existence of a sequence of moves between them. Crucially, flype moves don't change whether the diagram is alternating \noteblue{[figure out why this preserves alternatingness]}, and thus the Flyping Theorem provides a way to describe the equivalence of alternating knots via alternating diagrams only.

We also have ways of constructing new knots from existing ones. If we take a diagram, choose a tangle, and then perform a reflection (up to planar isotopy) of that tangle, either reflecting it left-right or up-down, or across one of the diagonals, the corresponding operation on the knot is known as \textit{mutation}, and the two knots called \textit{mutants}. Mutants are some of the hardest knots to distinguish, as many of their invariants are the same. An example of this is the Conway the Kinoshita-Terasaka knots, given in Fig.~\ref{fig:kinoshita-terasaka-mutants}.

\begin{figure}[hbt!]
	\centering
	\hspace*{\fill}
	\begin{subfigure}[b]{0.4 \textwidth}
		\centering
		\def\svgscale{0.25}
		\input{graphics/kinoshita-terasaka.pdf_tex}
		\caption{The Kinoshita-Terasaka Knot}
		\label{fig:kinoshita-terasaka-knot}
	\end{subfigure}
	\hspace*{\fill} \hspace*{\fill}	\hspace*{\fill}
	\begin{subfigure}[b]{0.4 \textwidth}
		\centering
		\def\svgscale{0.25}
		\input{graphics/conway.pdf_tex}
		\caption{The Conway Knot}
		\label{fig:conway-knot}
	\end{subfigure}
	\hspace*{\fill} 
	\caption{The Kinoshita-Terasaka mutants. These are the two non-trivial knots of lowest crossing number that have trivial Alexander polynomial. The disk of mutation is marked. The projections used for these diagrams were taken from \cite[Fig.~2.32]{the-knot-book}.}
	\label{fig:kinoshita-terasaka-mutants}
\end{figure}

To every knot diagram, we associate two graphs known as the \textit{Tait graphs}, shown in Fig.~\ref{fig:tait-example} and explained as follows. We interpret the knot diagram as a tetravalent planar graph that divides the plane into regions. An application of the Jordan curve theorem that it is possible to colour these regions two colours, black and white, such that adjacent regions are never the same colour. Such a colouring is called a \textit{checkerboard colouring}. Note that in a checkerboard colouring, regions that are diagonal to each other at crossings are necessarily the same colour.

For alternating knots, in a checkerboard colouring, all of the crossings in the knot will be of the same \textit{type}, either type $A$ or type $B$, as given in Fig.~\ref{fig:crossing-type}. The convention for alternating knots is to checkerboard colour such that all crossings are of type $A$. \notered{How is this formally defined? Are these actually called `type'? What else is there to say.}

\begin{figure}[hbt]
	\centering
	\hspace*{\fill}
	\begin{subfigure}[b]{0.4 \textwidth}
		\centering
		\def\svgscale{0.28}
		\input{graphics/checkerboard_type_a.pdf_tex}
		\caption{Type $A$.}
		\label{fig:type-a}
	\end{subfigure}
	\hspace*{\fill} \hspace*{\fill}	\hspace*{\fill}
	\begin{subfigure}[b]{0.4 \textwidth}
		\centering
		\def\svgscale{0.28}
		\input{graphics/checkerboard_type_b.pdf_tex}
		\caption{Type $B$.}
		\label{fig:type-b}
	\end{subfigure}
	\hspace*{\fill} 
	\caption{Crossing Type}
	\label{fig:crossing-type}
\end{figure}

To construct the black Tait graph $G_{B}(K)$, we place a vertex in every black region of the plane. For each crossing we draw an edge between the vertices corresponding to the black region on either side of the crossing, obtaining a planar graph, potentially with multiple edges. The white Tait graph $G_{W}(K)$ is constructed similarly from the vertices corresponding to white regions.

Either of these graphs retains enough information to construct the other, as they are \textit{planar duals}: that is, if every face in a planar diagram for $G_{B}(K)$ is replaced by a vertex, and edges between vertices in $G_{B}(K)$ replaced by edges between the faces they separate, then $G_{W}(K)$ has been constructed from $G_{B}(K)$ and vice-versa. For the planar dual of $G$, we write $G^{*}$; that is $G_{B}(K)^{*} = G_{H}(K)$. Because both planar graphs can be constructed from a single one, it is often only necessary to work with one of these graphs, and sometimes refer the \textit{the} Tait graph of a knot. Later we will examine the more general class of knots in thickened surfaces, for which this duality breaks down, and we will need both Tait graphs.

\begin{figure}[h]
	\centering
	\hspace*{\fill}
	\begin{subfigure}[b]{0.3 \textwidth}
		\centering
		\def\svgscale{0.2}
		\input{graphics/trefoil.pdf_tex}
		\caption{Trefoil Knot\dots}
		\label{fig:trefoil-blank}
	\end{subfigure}
	\hspace*{\fill}
	\begin{subfigure}[b]{0.3 \textwidth}
		\centering
		\def\svgscale{0.2}
		\input{graphics/trefoil_checker.pdf_tex}
		\caption{\dots with a checkerboard colouring \dots}
		\label{fig:trefoil-checker}
	\end{subfigure}
	\hspace*{\fill}
	\begin{subfigure}[b]{0.3 \textwidth}
		\centering
		\def\svgscale{0.2}
		\input{graphics/trefoil_checker_tait.pdf_tex}
		\caption{\dots and its Tait graph}
		\label{fig:trefoil-checker-tait}
	\end{subfigure}
	\hspace*{\fill}
	\caption{Producing the Tait graph(s) of the trefoil knot.}
	\label{fig:tait-example}
\end{figure}

If we have the Tait graph of an alternating diagram of a knot, we can reconstruct the knot diagram by placing a crossing on every edge of the Tait graph, as seen in Fig.~\ref{fig:trefoil-tait-with-crossings}. Connecting up the crossings by strands along the Tait graph, as in Fig.~\ref{fig:trefoil-resconstructed},	 reconstructs the knot diagram, up to planar isotopy.

\begin{figure}[h]
	\centering
	\hspace*{\fill}
	\begin{subfigure}[b]{0.3 \textwidth}
		\centering
		\def\svgscale{0.22}
		\input{graphics/trefoil_reconstruct_crossings.pdf_tex}
		\caption{Draw crossings on edges.}
		\label{fig:trefoil-tait-with-crossings}
	\end{subfigure}
	\hspace*{\fill}
	\begin{subfigure}[b]{0.3 \textwidth}
		\centering
		\def\svgscale{0.22}
		\input{graphics/trefoil_reconstructed.pdf_tex}
		\caption{Connect the crossings.}
		\label{fig:trefoil-resconstructed}
	\end{subfigure}
	\hspace*{\fill}
	\begin{subfigure}[b]{0.3 \textwidth}
		\centering
		\def\svgscale{0.2}
		\input{graphics/trefoil.pdf_tex}
		\caption{The reconstructed trefoil.}
		\label{fig:trefoil-resconstructed-isotopy}
	\end{subfigure}
	\hspace*{\fill}
	\caption{Reconstructing a knot diagram from its Tait graph.}
	\label{fig:tait-reconstruction-example}
\end{figure}

\section{The Lattice of Integer Flows}

For the rest of this chapter, we introduce the lattice of integer flows of a graph, then largely following \cite{lattices-graphs-mutation}, we show that the lattice of integer flows of the Tait graph is a complete mutation invariant of alternating knots. This means that it is both an invariant of alternating knots, and takes the same value on alternating knots if and only if they are mutants. We then talk about the equivalent formulation given in \cite{lattices-graphs-mutation} as the $d$-invariant.

A \textit{lattice} is a finitely generated abelian group $L$, equipped with an inner product 
\({\ip{\cdot}{\cdot}: L \times L \longrightarrow \R}\). All lattices in this text will be \textit{integral lattices}, meaning the inner product's image is contained within $\Z$. An \textit{isomorphism of lattices} is a bijection $\psi: L_{1} \longrightarrow L_{2}$ that preserves the inner product, that is, ${\ip{x}{y} = \ip{\psi(x)}{\psi(y)}}$ for all $x, y \in L$.

Let $G = (E, V)$ be a finite, connected graph (in which loops and multiple edges are allowed) with vertex set $V$ and edge set $E$. In particular, $G$ is a 1-dimensional CW-complex with boundary map $\partial: C_{1}(G) \longrightarrow C_{0}(G)$. In the edge basis of $C_{1}(G)$ and vertex basis of $C_{0}(G)$, this is a map $\Z^{E} \longrightarrow \Z^{V}$, and is represented by the $|V|\times|E|$ incidence matrix $D$ with entries given by
\[D_{ij} = \begin{cases}
	+1 & \text{if $e_{i}$ is oriented into $v_{j}$,}   \\
	-1 & \text{if $e_{i}$ is oriented out of $v_{j}$,} \\
	0  & \text{if $e_{i}$ is a loop at $v_{j}$ or is not incident to $v_{j}$.}
\end{cases}\]

The \textit{lattice of integer flows} of $G$ is the free abelian group $\Lambda(G) = \ker \partial$, along with the inner product induced by the Euclidean inner product on $\Z^{E}$. Equivalently, $\Lambda(G)$ is the first homology group of $G$, with inner products taken in $C_{1}(G)$. While the lattice $\Lambda(G)$ may depend on the orientation of the edges in $G$, its isomorphism class does not, as the isomorphism class of the homology group is independent of orientation, and the Euclidean inner product is preserved by sending an edge to its negation: $\ip{e_{i}}{e_{i}} =  \ip{-e_{i}}{-e_{i}} = 1$, and $\ip{e_{i}}{e_{j}} = \ip{-e_{i}}{e_{j}} = 0$ for $i \neq j$.

As an example, let $D$ be the diagram of the Trefoil knot given in Fig.~\ref{fig:tait-example} the Tait graphs of the trefoil knot   are shown in Fig.~\ref{fig:trefoil-abstract-tait}. We compute the lattice of integer flows of each of these graphs, $\Lambda(G_{B}(D))$ and $\Lambda(G_W(D))$. Rather than use this cumbersome notation, we shorten this to $\Lambda_{W}(D)$ or $\Lambda_{B}(D)$ for the lattice of integer flows of the Tait graph corresponding to the white/black regions, respectively. We orient the graphs arbitrarily to compute the lattice of integer flows, but as we will see the invariant produced is independent of this choice. We start with the more typical lattice of integer flows: that of the Tait graph corresponding to the white region (red in Fig.~\ref{fig:trefoil-abstract-tait}). There are two independent cycles, $-f_{1} + f_{2}$ and $-f_{2} + f_{3}$, so $\Lambda_{B}(3_{1})$ is generated as
\[\Lambda_{B}(D) = \langle -f_{1} + f_{2}, -f_{2} + f_{3} \rangle\,.\]
We are taking inner products in $\Z^{E}$ (that is, in the basis $(f_{1}, f_{2}, f_{3})$), so if $x$ and $y$ are coordinate vectors for $u$, and $v$, of two elements of $\Lambda_{B}(D)$ expressed in the basis $(-f_{1} + f_{2}, -f_{2} + f_{3})$, then
\[\langle u, v \rangle = x^{\top}\begin{bmatrix}
	2  & -1 \\
	-1 & 2
\end{bmatrix}y\,.\]
We call this matrix the \textit{pairing matrix} of the lattice $\Lambda_{B}(D)$ with respect to the basis ${(-f_{1} + f_{2}, -f_{2} + f_{3})}$.

\begin{figure}[hbt]
	\centering
	\def\svgscale{0.4}
	\input{graphics/trefoil_abstract_tait.pdf_tex}
	\caption{The Tait graphs $G_{B}(D)$ and $G_{W}(D)$.}
	\label{fig:trefoil-abstract-tait}
\end{figure}

The pairing matrix of a lattice completely describes the lattice, however the pairing matrix is not completely determined by the lattice. Instead, the lattice only determines a pairing matrix up to unimodular congruence: two matrices $P$ and $P'$ are \textit{unimodular congruent} if there exists a matrix $A$ with $\det(A) = 1$, such that $P = A^{\top}P'A$. If $P$ is a pairing matrix of a lattice with respect to the basis $\mathcal{B}$ (with basis matrix $B$), and $P'$ is unimodular congruent to $P$ as above, then:
\begin{align*}
	\langle u, v \rangle & = [u]_{\mathcal{B}}^{\top} P [v]_{\mathcal{B}}                    \\
	                     & = B[u]_{\mathcal{E}}^{\top} P B[v]_{\mathcal{E}}                  \\
	                     & = B[u]_{\mathcal{E}}^{\top} A^{\top} P' A B[v]_{\mathcal{E}}      \\
	                     & = \left( AB[u]_{\mathcal{E}} \right)^{\top} P' AB[v]_{\mathcal{E}} \\
	                     & = \left( [u]_{\mathcal{AB}} \right)^{\top} P' [v]_{\mathcal{AB}}\,,
\end{align*}
so $P'$ is also a pairing matrix of the lattice with respect to the basis whose basis matrix is $AB$.


Continuing with our example, there is a single independent cycle in $G_{W}(D)$, it being $e_{1} + e_{2} + e_{3}$. Hence the pairing matrix of $\Lambda_{W}(D)$ with respect to $( e_{1} + e_{2} + e_{3})$ is $P = [3]$.

We now investigate some properties of the lattice of integer flows. Since the 1970's, many graph theorists have studied exactly how much information about the graph $G$ is retained by $\Lambda(G)$ \parencite{lattice-of-flows-cuts, torelli-for-graphs-tropical-curves, lattice-of-flows-regular-matroid}. In the next paragraph we see that some information about $G$ is lost.

We call and edge $e$ of a graph $G$ a \textit{bridge} if the removal of $G$ from $e$ disconnects $G$, and we say a graph $G$ is \textit{$2$-edge-connected} if $G$ has no bridges. The lattice of integer flows of $G$ is blind to its bridges. We define $G_{\bullet}$ as $G$ with its bridges contracted: that is, for each bridge $e$, we remove $e$ and identify the adjacent vertices, then $\Lambda(G_{\bullet}) = \Lambda(G)$. This is because the cycles of $G$, and therefore the homology of $G$, do not see bridges.

A \textit{$2$-isomorphism} between two graphs $G = (E, V)$ and $G' = (E', V')$ is a bijection \({\psi: E \longrightarrow E'}\) which preserves cycles, i.e. $\partial(e_{i} + \cdots + e_{j}) = 0$ if and only if $\partial\left(\psi(e_{i}) + \cdots + \psi(e_{j})\right) = 0$. It is well established that $\Lambda(G)$ is a $2$-isomorphism invariant \parencite{lattice-of-flows-cuts}. Recent work by Su-Wagner and Caporaso-Viviani have also shown that for $2$-edge-connected graphs, $\Lambda(G)$ is a complete $2$-isomorphism invariant \cites[Theorem 3.1.1]{torelli-for-graphs-tropical-curves}[Theorem 1]{lattice-of-flows-regular-matroid}. This is stated in the following theorem, which derives its name from a similar result in the context of algebraic geometry.


\begin{theorem}[Torelli Theorem for Graphs]
For two $2$-edge-connected graphs $G$ and $G'$, $\Lambda(G) \cong \Lambda(G')$ if and only if $G$ and $G'$ are $2$-isomorphic.
\end{theorem}

Two graphs $G$ and $G'$ are related by a \textit{Whitney flip} if it is possible to find two disjoint graphs $\Gamma_{1}$, with distinguished vertices $u_{1}$ and $v_{1}$ and $\Gamma_{2}$ with distinguished vertices $u_{2}$ and $v_{2}$, such that the identifications $u_{1} = u_{2} = u$ and $v_{1} = v_{2} = v$ form $G$, and the identifications $u_{1} = v_{2} = u'$ and $v_{1} = u_{2} = v'$ form $G'$. An example of graphs related by a Whitney flip is given in Fig.~\ref{fig:whitney_flip}.

\begin{figure}[hbt!]
	\centering
	\def\svgscale{0.5}
	\input{graphics/whitney_flip.pdf_tex}
	
	\caption{An example of a Whitney flip.}
	\label{fig:whitney_flip}
\end{figure}

It is clear that sequences of Whitney flips only ever transform graphs within their $2$-isomorphism class, as cycles map to cycles. From \cite{2-isomorphic-graphs}, we have the important converse; a Reidemeister-like theorem for $2$-isomorphic graphs.

\begin{theorem}[Whitney's Theorem]
Two graphs $G$ and $G'$ are $2$-isomorphic if-and-only-if there is a sequence of Whitney flips relating $G$ to $G'$.
\end{theorem}

\notered{Rephrase} It can be easily shown that flype moves correspond to Whitney flips of the Tait graphs of the knot \notered{[figure]}. With this and Whitney's theorem, we obtain that $\Lambda_{B}(D)$ is an invariant of alternating knots, and $\Lambda_{W}(D)$ is an equivalent invariant.

\begin{theorem}
Let $K$ be alternating, and let $D$, $E$ be alternating diagrams for $K$. Then $\Lambda_{B}(D) \cong \Lambda_{B}(E)$. Furthermore, $\Lambda_{B}(D) \cong \Lambda_{B}(E)$ if and only if $\Lambda_{W}(D) \cong \Lambda_{W}(E)$.
\end{theorem}

\begin{proof}
All non-reduced alternating diagrams of $K$ reduce to an alternating diagram for $K$ without change to the lattice of integer flows of their Tait graphs, so we may assume that $D$ and $E$ are alternating reduced diagrams for $K$. 

First we consider the case where $K$ is a prime knot. By the flyping theorem, $D$ and $E$ are related by a sequence of flypes. As flypes induce Whitney flips of the Tait graph, and by Whitney's theorem, $G_{B}(D)$ and $G_{B}(E)$ are $2$-isomorphic. As reduced diagrams of alternating knots are $2$-edge-connected \notered{[cite?]}, the Torelli Theorem for Graphs applies, and so $\Lambda_{B}(E) \cong \Lambda_{B}(E)$.

We also consider where $K$ is composite. By Schubert's Theorem \notered{[cite]}, $K = K_{1} \# K_{2} \# \cdots \# K_{n}$ \notered{[define $\#$]}, with each $K_{i}$ prime for $i \in \{1, \cdots, n\}$, and this decomposition is unique up to permutation. A fact about alternating knots is that a knot is alternating if and only if its prime decomposition is of alternating prime knots \notered{[find reference; turn into lemma above]}, hence each $K_{i}$ is alternating, for $i \in \{1, \cdots, n\}$. A theorem of Menasco \notered{[cite]} is that a reduced alternating diagram represents a prime knot if and only if the diagram itself is prime \notered{[define above: but the definition is what one would think]}. Since $K$ is composite, $D$ is a composite diagram, so \[D = D_{1} \# D_{2} \# \cdots \# D_{m}\] for some $m > 1$. But by the unique prime decomposition, we may assume that $m = n$ and $D_{i}$ is a diagram for $K_{i}$ for $i \in \{1, \cdots, n\}$ (if this is not the case, then since $D$ is a diagram for $K$, $K$ has another prime decomposition, a contradiction). Furthermore, as every reduced diagram of a prime alternating knot is alternating \notered{[cite; \footnote{this is actually two results: Theorem (Kauffman, Murasugi, Thistlethwaite, 1987): ``Any reduced diagram of an alternating link has the fewest possible crossings'' and Theorem (Kauffman, Murasugi, Thistlethwaite, 1987): ``Every minimal crossing diagram of a prime alternating link is alternating''}]}, each $D_{i}$ is an alternating diagram. Without loss of generality, the same is true for the \[E = E_{1} \# E_{2} \# \cdots \# E_{n}\] with $E_{i}$ with $i \in \{1, \cdots, n\}$. For all such $i$, we have that $K_{i}$ is alternating and prime, and $E_{i}$ and $D_{i}$ are alternating diagrams for $K_{i}$. The hypothesis in the first case is satisfied, so $\Lambda_{B}(D_{i}) \cong \Lambda_{B}(E_{i})$ for each $i \in \{1, \cdots, n\}$. Finally, since the Tait graph of a connected sum of knots is the wedge sum at a vertex of the Tait graphs of the knots, we have
\begin{align*}
	\Lambda_{B}(D) & = \Lambda_{B}(D_{1}) \oplus \Lambda_{B}(D_{2}) \oplus \cdots \oplus \Lambda_{B}(D_{n}) \\
	               & = \Lambda_{B}(E_{1}) \oplus \Lambda_{B}(E_{2}) \oplus \cdots \oplus \Lambda_{B}(E_{n})\\
	               & = \Lambda_{B}(E)\\
\end{align*}
\end{proof}

This shows that $\Lambda_{B}(D)$ and $\Lambda_{B}(D)$ is a well-defined function on alternating knots, and henceforth we may write $\Lambda_{B}(K)$ and $\Lambda_{W}(K)$, so long as $K$ is alternating.

Greene examines the effect of both flype moves and mutation on the Tait graph \cite[Lemma 4.5]{lattices-graphs-mutation}:

\begin{lemma}[Greene]
Depending on the tangle to be mutated, a mutation of $D$ either:
\begin{itemize}
\item affects only the planar embedding of the Tait graph, leaving its graph structure unchanged, or
\item induces a Whitney flip in the Tait graph.
\end{itemize}
Conversely, a Whitney flip in the Tait graph induces a mutation in $D$.
\end{lemma}

This lemma is useful in proving one of Greene's key results: that the lattice of integer flows of the Tait graph is a complete mutation invariant of alternating knots \cite[Proposition 4.4]{lattices-graphs-mutation}.

\begin{theorem}[Greene]
Let $K_{1}$ and $K_{2}$ be alternating knots. Then, $\Lambda_{B}(K_{1}) \cong \Lambda_{B}(K_{2})$ and $\Lambda_{W}(K_{1}) \cong \Lambda_{W}(K_{2})$ if an only if $K_{1}$ and $K_{2}$ are mutants.
\end{theorem}

\begin{proof}
Let $K_{1}$ and $K_{2}$ be alternating virtual knots. For the forward implication, suppose that $K_{1}$ and $K_{2}$ are mutants, so there is a sequence of mutations from a diagram of $K_{1}$ to a diagram of $K_{2}$. By the lemma above, their Tait graphs $G_{B}(K_{1})$ and $G_{B}(K_{2})$  are related by Whitney flips (as are their respective white Tait graphs). By Whitney's theorem, each of these pairs of graphs are $2$-isomorphic. Hence $\Lambda_{B}(K_{1}) \cong \Lambda_{B}(K_{2})$ and $\Lambda_{W}(K_{1}) \cong \Lambda_{W}(K_{2})$.

For the reverse implication, take two alternating virtual knots such that $\Lambda_{B}(K_{1}) \cong \Lambda_{B}(K_{2})$ and $\Lambda_{W}(K_{1}) \cong \Lambda_{W}(K_{2})$. We may assume them to be reduced, as reduction of a knot does not alter the lattice of integer flows of its Tait graphs. As they are reduced, they are now $2$-edge-connected. So, $G_{B}(K_{1})$ and $G_{B}(K_{2})$ are $2$-isomorphic, and likewise for the Tait graphs corresponding to the white regions of the checkerboard colouring. By Whitney's theorem, there is a sequence of Whitney flips between each of these pairs, which by the lemma above corresponds to a sequence of mutations between $K_{1}$ and $K_{2}$, so they are mutants. \notered{For the reverse implication is it necessary to consider both Tait graphs? What happens if only one pair is isomorphic? Is this possible: i.e. for duals does $\Lambda(G) \cong \Lambda(H) \implies \Lambda(G^{*})$ $\cong$ $\Lambda(H^{*})$?}
\end{proof}

\section{\notegreen{The $d$-invariant and Heegard-Floer Homology}}
\notegreen{Show that can be compressed into the $d$-invariant, and nothing is lost.}

\chapter{Virtual Knots}

\epigraph{\itshape I thought this was the end\\But no my friends, this is when\\We get to do it all again...}{--- The Muppets}
\epigraph{\itshape ...and now it's virtual insanity.}{--- Jamiroquai}


We now introduce the exciting and relatively new theory of virtual knots. Virtual knots are a generalisation of knots, and there are many different equivalent formulations of them. We start with the most geometric of the formulations, but we also present a combinatorial and computational definition later.


\section{Knots in Thickened Surfaces}

\textit{Classical} knots, a term which refers specifically to the kind of knots we have introduced prior to this chapter, have diagrams in the plane, $\R^{2}$, but really they have an extra dimension of `thickness', encoded in the diagram by the under- and over- crossings. Hence we think of knots as embeddings in $\R^{3}$. However we didn't really need a whole $\R$'s worth of extra space. We could easily think of classical knots as living in a thickened plane, $\R^{2} \times I$ where $I$ is the unit interval $[0, 1]$. Thinking of knots as embeddings in $\R^{2} \times I$, it becomes natural to ask: what if we replace the plane by another surface; can we have diagrams on other surfaces $\Sigma$ and therefore knots in \textit{thickened surfaces} $\Sigma \times I$? The answer to these questions is yes, and virtual knots are one such generalisation.

In the context of virtual knots, all surfaces of relevance are closed and orientable, with the exception of the plane. However, knots diagrams on the plane are equivalent to knot diagrams on the sphere. The only extra move allowed by this compactification is that a strand on one side of a knot diagram can be taken over to the other \notered{[figure]} by isotopy of moving it around the back side of the sphere. However this was already allowed on the plane by a sequence of R\textit{III} moves.  The classification theorem for compact, orientable surfaces is the following.

\begin{theorem}[Classification of compact, orientable surfaces]
Each connected component of a compact, orientable surface is homeomorphic to:
\begin{itemize}
	\item the sphere, or
	\item a connected sum of $g$ tori, for $g \geq 1$.
\end{itemize}
\end{theorem}
Hence there is a bijection between connected components of compact, orientable surfaces given by the \textit{genus}, $g$ of the surface, the number of handles.

We now follow the work of Kuperberg \cite{what-is-a-virtual-link} and Carter-Kamada-Saito \cite{stable-equivalence-virtual-cobordisms} and give the geometric definition of virtual knots. A \textit{surface knot diagram} on $\Sigma$ is the analogue of a classical knot diagram, but drawn on a general closed, oriented, connected surface, $\Sigma$ no longer necessarily the plane. To represent diagrams in this section we use the letter $P$, for `projection', another common word for `diagram', as $D$ will be reserved for disks. The equivalence relation on surface knot diagrams is given by the Reidemeister moves and surface isotopy (the surface-analogue of planar isotopy) on $\Sigma$.

We can define knots on thickened surfaces so that they relate to surface knot diagrams, in the same way knots and knot diagrams are related in the classical context. A \textit{knot in a thickened surface} $\Sigma \times I$ is an embedding $K:S^{1} \lhook\joinrel\longrightarrow \Sigma \times I$ up to ambient isotopy in $\Sigma \times I$.

To define virtual knots in thickened surfaces of any genus under another equivalence relation: stable equivalence, defined based on the following two operations. We give the definitions on the level of surface knot diagrams, but note that they generalise to knots in thickened surfaces accounting for the extra factor of $I$. The operation of \textit{stabilisation} consists of finding two disks $D_{1}$ and $D_{2}$ in $\Sigma$ that do not intersect $P$. We then remove $D_{1}$ and $D_{2}$ from $\Sigma$ and glue a handle whose boundary is $D_{1} \cup D_{2}$. Intuitively, stabilisation is `adding a handle' to $\Sigma$, and any newly added handle does not interact with $P$. The reverse operation \textit{destabilisation} removes a handle. Performing this operation, we take a cylinder $Y$ that does not intersect $P$, and such that the circle that $Y$ deformation-retracts to is not null-homologous, removes it, and cap both resulting boundary circles. The point of stable equivalence is to identify two knot diagrams $P_{1}$ and $P_{2}$ that would otherwise be identical, except for living on surfaces $\Sigma_{1}$ and $\Sigma_{2}$ of different genus $g_{1}$ and $g_{2}$; for one surface must have a greater genus than the other $g_{1} < g_{2}$. But since the diagrams are identical, there must be at least $g_{2} - g_{1}$ handles in the diagram on $\Sigma_{2}$ that are unnecessary and don't interact in any way with the diagram $P_{2}$.

Hence we define a \textit{virtual knot} as an equivalence class of knots on thickened surfaces under stable equivalence. Each virtual knot has a minimum genus surface, in which all of the handles interact with the the knot, and this is known as its virtual genus.

The virtual knots with virtual genus $g_{v} = 0$ correspond to the classical knots, and the virtual knots with virtual genus $g_{v} < 0$ are \textit{strictly} virtual. An example of a strictly virtual knot is given in Fig.~\ref{fig:not-checkerboard-colourable}.

\begin{figure}[hbt]
	\centering
	\def\svgscale{0.35}
	\input{graphics/2-1_vknot.pdf_tex}
	\caption{Virtual knot $2_{1}$, a strictly virtual knot that is not checkerboard colourable. This knot is drawn on the gluing diagram of the torus, $\T^{2}$, its thickened surface of minimal genus.}
	\label{fig:not-checkerboard-colourable}
\end{figure}

Instead of drawing virtual knots directly on compact orientable surfaces, for genus $g \geq 1$, we often draw them on the gluing diagrams of those surfaces. From a textbook theorem of algebraic topology \parencite{algebraic-topology}, the compact orientable surface of genus $g$ is obtained from the $4g$-gon by gluing around the polygon with the pattern $aba^{-1}b^{-1}$,~$cdc^{-1}d^{-1}$,~$\cdots$ continuing on for $g$ iterations, or until all edges have been glued. Figures~\ref{fig:not-checkerboard-colourable}~and~\ref{fig:counterexamples} are examples of this, with the surfaces being the torus, $\T^{2}$ and the compact orientable surface of genus 2, $2\T^{2}$, respectively.

There is no fast algorithm to compute the virtual genus of a virtual knot, however the following theorem from \cite{parity-and-projection} does allow some progress:
\begin{theorem}
	If a diagram of a knot $K$ in a thickened surface $\Sigma \times I$ contains the minimal number of crossings across all diagrams in thickened surfaces, then $\Sigma$ is the minimal genus surface that can support $K$. That is, the genus of $\Sigma$ is the knot's virtual genus.
\end{theorem}
Applying this theorem to the classification of virtual knot diagrams up to $6$ crossings by Jeremy Green \cite{virtual-knot-table} we can determine the virtual genus of virtual knots up to $6$ crossings. \noteblue{More about this computation will be explained in Chapters 4 and 5.}

\section{Knots with Virtual Crossings}

The original formulation of virtual knots (and their discovery) is due to Kauffman in 1996 \cite{virtual-knot-theory}. This formulation can be related to the formulation of equivalence classes of diagrams on $\Sigma$ by projecting the surface $\Sigma$ the onto the plane. Doing this creates two types of crossings. Those that did actually come from a crossing on $\Sigma$, we call \textit{classical crossings}, and they have the usual over- and under- strands as determined by the projection. Those that did not exist on $\Sigma$ but rather are an artefact of the projection we call \textit{virtual crossings}. For strictly virtual knots these virtual crossings will be necessary, as the $4$-valent graph that the knot represents is not planar.

\begin{figure}[hbt]
	\centering
	\hspace*{\fill}
	\begin{subfigure}[b]{0.35 \textwidth}
		\centering
		\def\svgscale{0.22}
		\input{graphics/virtual_reidemeister_1.pdf_tex}
		\caption{\textit{VR}I}
	\end{subfigure}
	\hspace*{\fill}
	\begin{subfigure}[b]{0.35 \textwidth}
		\centering
		\def\svgscale{0.22}
		\input{graphics/virtual_reidemeister_2.pdf_tex}
		\caption{\textit{VR}II}
	\end{subfigure}
	\hspace*{\fill}
	\\
	\hspace*{\fill}
	\begin{subfigure}[b]{0.35 \textwidth}
		\centering
		\def\svgscale{0.22}
		\input{graphics/virtual_reidemeister_3.pdf_tex}
		\caption{\textit{VR}III}
	\end{subfigure}
	\hspace*{\fill}
	\begin{subfigure}[b]{0.35 \textwidth}
		\centering
		\def\svgscale{0.22}
		\input{graphics/virtual_reidemeister_4.pdf_tex}
		\caption{\textit{VR}IV}
	\end{subfigure}
	\hspace*{\fill}
	\caption{The four additional virtual Reidemeister moves.}
	\label{fig:virtual-reidemeister-moves}
\end{figure}

The relevant equivalence relation on diagrams with virtual crossings are not hard to deduce. We have the three Reidemeister moves which still hold between classical crossings, three corresponding moves similar to the Reidemeister moves but with all classical crossings replaced by virtual crossings, and finally a `triangle move' that moves a `virtual strand' through a crossing.

There is yet another interpretation of virtual knots that we explore in this paper, a computational definition that is integral to computing invariants of virtual knots. We will explore this later a later chapter.

\section{Mutation and the Tait Graph for Virtual Knots}
Having introduced virtual knots, the rest of this chapter will be focussed on constructing a generalisation of the $d$-invariant of \cite{lattices-graphs-mutation}, or rather, the equivalent lattices of integer flows of the Tait graphs of the knot. First we must introduce the tools we used to define this invariant in Chapter 1 into this new virtual setting. We focus on the definition of virtual knots as knots on thickened surfaces under stable equivalence.

There are two types of mutation of virtual knots: disk mutation and surface mutation. \textit{Disk mutation} is directly analogous to mutation of classical knots. We take a disk $D \subseteq \Sigma$ that contains a tangle and flip or rotate it, and the resulting knot is a disk mutant of the original. This is in contrast to \textit{surface mutation}, in which the chosen subset need only have circular boundary, but could contain handles. Surface mutation is a more invasive operation and we do not consider it in the present text. However, future work may lie in investigating the equivalence classes generated by surface mutation. For the rest of this text, mutation, in the virtual context, refers to disk mutation.

In the classical case, all knots were checkerboard colourable, so the Tait graph could always be produced. Furthermore, only one Tait graph was necessary to encode an alternating knot, as Tait the two graphs were always planar duals to each other. In the virtual case, neither of these facts hold.

If we take a knot in a thickened surface $\Sigma$ that is checkerboard colourable, such as that in Fig.~\ref{fig:4-105-vknot}, it is always possible to put it on a new surface $\Sigma'$ on which it will not be checkerboard colourable. We do this in Fig.~\ref{fig:4-105-vknot-with-handle} by adding a handle between a black region and a white one, unifying them. The resulting region now needs to be both black and white, so the knot on $\Sigma'$ is not checkerboard colourable. Hence when trying to checkerboard colour virtual knots, the surface needs to be taken into account.

\begin{figure}[hbt]
	\centering
	\hspace*{\fill}
	\begin{subfigure}[b]{0.4 \textwidth}
		\centering
		\def\svgscale{0.35}
		\input{graphics/4-105_vknot_checker.pdf_tex}
		\caption{Virtual knot $4_{105}$}
		\label{fig:4-105-vknot}
	\end{subfigure}
	\hspace*{\fill}	\hspace*{\fill}	\hspace*{\fill}
	\begin{subfigure}[b]{0.4 \textwidth}
		\centering
		\def\svgscale{0.35}
		\input{graphics/4-105_vknot_handle.pdf_tex}
		\caption{A handle is added}
		\label{fig:4-105-vknot-with-handle}
	\end{subfigure}
	\hspace*{\fill} 
	\caption{Adding a handle can disrupt checkerboard colourability.}
	\label{fig:adding-handle-to-4-105}
\end{figure}

We say a knot in a thickened surface is \textit{checkerboard colourable} if there is a checkerboard colourable surface knot diagram that represents it. We have the following classification of checkerboard colourable knots in thickened surfaces \cite{minimal-diagrams-surface-links}.

\begin{theorem}
Given a knot in a thickened surface, $K \subset \Sigma \times I$, the following are equivalent:
	\begin{enumerate}[(i)]
	\item $K$ is checkerboard colourable,
	\item $K$ is the boundary of an unoriented spanning surface $F \subset \Sigma \times I$,
	\item $[K] = 0$ in the homology group $H_{1}(\Sigma \times I; \Z_{2})$.
	\end{enumerate} 
\end{theorem}

Likewise, we say a virtual knot is \textit{checkerboard colourable} if there is a representative that is checkerboard colourable (as a knot in a thickened surface). Note that while Fig.~\ref{fig:4-105-vknot-with-handle} is indeed checkerboard colourable as a virtual knot, it is not checkerboard colourable as a knot in a thickened surface.

For the purposes of generalising the $d$-invariant, we need only concern ourselves with alternating virtual knots -- those virtual knots that can be represented by alternating virtual knot diagrams. Fortunately the following theorem tells us that we will not run into any trouble \cite{jones-polynomial-checkerboard-colourable}.

\begin{theorem}[Kamada's Lemma]
	Every alternating virtual knot is checkerboard colourable. \notered{Check with Hans the correct statement of this lemma. I think we need cellularly embedded.}
\end{theorem}

\notered{Do we want to put the virtual flyping theorem here? Because without it we don't have flype moves so we don't have invariants of alternating virtual knots.}

Hence for all alternating surface knot diagrams we can define the Tait graph. 

\section{The Virtual $d$-invariant}
\notegreen{Invariant due to Virtual Flyping (Kindred). Mutation invariant comes in a similar way. Complete mutation invariant? we conjecture not.}

\chapter{Gordon-Litherland Linking Form}

\epigraph{\itshape God created the knots. All else in topology is the work of man.}{--- Leopold Kronecker, modified by Dror Bar-Natan}

In particular, we turn our attention to the work of two men: Cameron McA. Gordon and Richard A. Litherland who did \notered{something} \cite{signature-of-a-link}.

Then Hans et. al. made it virtual. It has this interpretation something something double branched covers \cite{gordon-litherland-pairing-thickened-surfaces}.

\chapter{Gauss Codes and Knot Algorithms}
\notered{Work in pages already written about the algorithm to find the Tait graph.}


\chapter{Computing Mock Seifert Matrices}

\begin{figure}[hbt!]
	\centering
	\hspace*{\fill}
	\begin{subfigure}[b]{0.4 \textwidth}
		\centering
		\def\svgscale{0.35}
		\input{graphics/6-90101_vknot.pdf_tex}
		\caption{Virtual knot $6_{90101}$}
		\label{fig:6-90101_vknot}
	\end{subfigure}
	\hspace*{\fill}	\hspace*{\fill}	\hspace*{\fill}
	\begin{subfigure}[b]{0.4 \textwidth}
		\centering
		\def\svgscale{0.35}
		\input{graphics/6-90124_vknot.pdf_tex}
		\caption{Virtual knot $6_{90124}$}
		\label{fig:6-90124_vknot}
	\end{subfigure}
	\hspace*{\fill} 
	\caption{write}
	\label{fig:counterexamples}
\end{figure}

From the tables produced, it was noticed that the Kobayashi invariant \cite{new-invariant-under-congruence}, defined as
\[\kob A = \operatorname{tr}(A^{\top}A^{-1})\]
always satisfied a relation involving the coefficients of the Alexander polynomial
\[\Delta_{A}(t) \coloneq \det(tA - A^{\top})\] \cite{mock-seifert-matrices}.

\begin{proposition}
The Kobayashi invariant satisfies
\[\kob A = -\dfrac{a_{1}}{a_{0}}\]
where $a_{i}$ is the $i$th coefficient of the Alexander polynomial $\Delta_{A}(t)$.
\end{proposition}
\begin{proof}
The matrix $A$, being a mock Seifert matrix, has odd determinant \cite{mock-seifert-matrices} and is therefore invertible. Hence by the multiplicity of determinants,
\[\det(tA - A^{\top}) = \det(A)\det(tI - A^{\top}A^{-1})\,.\]
The last factor here is the characteristic polynomial of $A^{\top}A^{-1}$. It is a well known fact that the zeroth and first coefficients of the characteristic polynomial of $M$, are $\det(M)$ and $-\operatorname{tr}(M)$ respectively. Hence,
\begin{align*}
-\dfrac{a_{1}}{a_{0}}	& = \dfrac{\operatorname{tr}(A^{\top}A^{-1})}{\det(A^{\top}A^{-1})}\,,
\end{align*}
but as $\det(A^{\top}A^{-1}) = \det(A^{\top})\det(A^{-1}) = \det(A)\det(A^{-1}) = 1$, we have
\[-\dfrac{a_{1}}{a_{0}} = \operatorname{tr}(A^{\top}A^{-1})\]
as required.
\end{proof}

This means that the Kobayashi invariant is weaker than the Alexander polynomial.


\newpage
\printbibliography[title=References]


\appendix
\titleformat{\chapter}[block]
  {\normalfont\Large\bfseries}{Appendix \thechapter}{1em}{\Large}
\titlespacing*{\chapter}{0pt}{40pt}{30pt}

\chapter{Algorithm}


\end{document}